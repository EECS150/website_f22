%texexptitled======================================================================
% lab1-gcd
%-----------------------------------------------------------------------
%

\documentclass[11pt]{article}

% Package includes

\usepackage[table]{xcolor}
\usepackage{graphicx}
\usepackage{color}
\usepackage{comment}
\usepackage{multirow}
\usepackage{askmaps}
\usepackage{karnaughmap}
\usepackage{amssymb}
\usepackage{amsmath}
\usepackage{gensymb}
\usepackage{arydshln}

% Wrap long URLs with hyphens
\PassOptionsToPackage{hyphens}{url}\usepackage{hyperref}
\usepackage[T1]{fontenc} %Get better tilde
\usepackage{lmodern} %Get better tilde
\usepackage{pdftexcmds}
\usepackage{pdfpages}
\usepackage{upquote}
\usepackage{textcomp}
\usepackage{minted}
%\usepackage[listings]{tcolorbox}
\usepackage{enumerate}
\usepackage{enumitem}
\usepackage{mathtools}
\usepackage{tikz}
\usepackage{circuitikz}
\usetikzlibrary{arrows, positioning, shapes.geometric, circuits.logic.US}
\tikzstyle{line}=[draw]
\tikzstyle{arrow}=[draw, -latex]
\usepackage[framemethod=tikz]{mdframed} %For easy shading of solutions

\usepackage{datetime2} % can get the time the PDF was generated
\usepackage{fancyhdr} % can set the footer


\usepackage{enumitem} % I'm not sure why I needed this
\usepackage[symbol]{footmisc} % use symbolic footnote symbols, except it doesn't work
\renewcommand{\thefootnote}{\fnsymbol{footnote}}

\DeclarePairedDelimiter{\ceil}{\Big\lceil}{\Big\rceil}

%\tcbset{
%texexp/.style={colframe=black, colback=lightgray!15,
%         coltitle=white,
%         fonttitle=\small\sffamily\bfseries, fontupper=\small, fontlower=\small},
%     example/.style 2 args={texexp,
%title={Question \thetcbcounter: #1},label={#2}},
%}
%
%\newtcolorbox{texexp}[1]{texexp}
%\newtcolorbox[auto counter]{texexptitled}[3][]{%
%example={#2}{#3},#1}

\setlength{\topmargin}{-0.5in}
\setlength{\textheight}{9in}
\setlength{\oddsidemargin}{0in}
\setlength{\evensidemargin}{0in}
\setlength{\textwidth}{6.5in}

% Useful macros

\newcommand{\note}[1]{{\bf [ NOTE: #1 ]}}
\newcommand{\fixme}[1]{{\bf [ FIXME: #1 ]}}
\newcommand{\wunits}[2]{\mbox{#1\,#2}}
\newcommand{\um}{\mbox{$\mu$m}}
\newcommand{\mm}{\mbox{mm}}
\newcommand{\nm}{\mbox{nm}}
\newcommand{\xum}[1]{\wunits{#1}{\um}}
\newcommand{\by}[2]{\mbox{#1$\times$#2}}
\newcommand{\byby}[3]{\mbox{#1$\times$#2$\times$#3}}
\usepackage{siunitx}

\newenvironment{tightlist}
{\begin{itemize}
 \setlength{\parsep}{0pt}
 \setlength{\itemsep}{-2pt}}
{\end{itemize}}

\newenvironment{titledtightlist}[1]
{\noindent
 ~~\textbf{#1}
 \begin{itemize}
 \setlength{\parsep}{0pt}
 \setlength{\itemsep}{-2pt}}
{\end{itemize}}

% Change spacing before and after section headers

\makeatletter
\renewcommand{\section}
{\@startsection {section}{1}{0pt}
 {-2ex}
 {1ex}
 {\bfseries\Large}}
\makeatother

\makeatletter
\renewcommand{\subsection}
{\@startsection {subsection}{1}{0pt}
 {-1ex}
 {0.5ex}
 {\bfseries\normalsize}}
\makeatother

% Reduce likelihood of a single line at the top/bottom of page

\clubpenalty=2000
\widowpenalty=2000

% Other commands and parameters

\pagestyle{fancy}
\setlength{\parindent}{0in}
\setlength{\parskip}{10pt}

% Commands for register format figures.

\newcommand{\instbit}[1]{\mbox{\scriptsize #1}}
\newcommand{\instbitrange}[2]{\instbit{#1} \hfill \instbit{#2}}

\newif\ifsolution
\newenvironment{solution}
%{\begin{mdframed}[backgroundcolor=gray!10, frametitle={Solution:}, frametitlebackgroundcolor=black, frametitlefont={\color{white}}]} %see http://mirror.utexas.edu/ctan/macros/latex/contrib/mdframed/mdframed.pdf
{\begin{mdframed}[frametitle={Solution:}, frametitlebackgroundcolor=black, frametitlefont={\color{white}}]} 
{\end{mdframed}}
%    {\color{red}}
%    {\color{black}}

\graphicspath{{./figs/}}

%Setting the fancy header & footer
\lhead{\assignmentname}
\rhead{\thepage}
\cfoot{\versionstamp}

%Display the version stamp on the 1st page but do not display the header
\fancypagestyle{plain}{%
\fancyhf{} % clear all header and footer fields
\fancyfoot[C]{\versionstamp} % except the center
\renewcommand{\headrulewidth}{0pt}
}

%Creating Column types for Fixed width (based on https://tex.stackexchange.com/questions/12703/how-to-create-fixed-width-table-columns-with-text-raggedright-centered-raggedlef)
\newcolumntype{L}[1]{>{\raggedright\let\newline\\\arraybackslash\hspace{0pt}}m{#1}}
\newcolumntype{C}[1]{>{\centering\let\newline\\\arraybackslash\hspace{0pt}}m{#1}}
\newcolumntype{R}[1]{>{\raggedleft\let\newline\\\arraybackslash\hspace{0pt}}m{#1}}

%Change Section Title Format (from https://tex.stackexchange.com/questions/245089/how-to-change-the-section-title-and-its-arrangement-in-a-latex-document)
%\renewcommand{\thesection}{\Roman{section}}
\usepackage{titlesec}
\titleformat{\section}
{\normalfont\Large\bfseries}{Problem~\thesection:}{1ex}{}

%-----------------------------------------------------------------------
% Document
%-----------------------------------------------------------------------


\begin{document}
\def\PYZsq{\textquotesingle}

\newcommand{\assignmentname}{EECS 151/251A Homework 1}
\newcommand{\versionstamp}{Version: 1 - \DTMnow}

\title{\vspace{-0.4in}\Large \bf \assignmentname \vspace{-0.1in}}
\author{Due Thursday, September 8\textsuperscript{th}, 2022 11:59PM}

\date{}
\maketitle

\thispagestyle{plain}

\section*{Problem 1: Dennard Scaling}
Assuming perfect Dennard Scaling. Given a processor that runs at 5MHz and dissipates 5W, what would the power and performance be in the next technology node if transistors are 1.25x smaller? Remember units!

\begin{solution}
$\kappa = 1.25$

Delay improves by 1.33, so the max frequency can be $5 \cdot 1.33 = \SI{6.66}{\giga\hertz}$.

Power density remains the same, but power dissipation scales with $s^2$, so power dissipation is $40 \cdot (0.75)^2 = \SI{22.5}{\watt}$
\end{solution}

\section*{Problem 2: What's that Circuit!}
\begin{enumerate}[label=(\alph*)]

\begin{figure}[H]
\centering
\includegraphics[width=0.6\textwidth]{figs/mux.png}
\caption{Circuit 1 for parts (a) (b) (c)}
\end{figure}

\item Write out and simplify the Boolean expression for circuit 1.

\item Write out the full truth table for circuit 1 based on the Boolean expression you found in part (a).

\item What is circuit 1 called?
\pagebreak

\begin{solution}
\begin{enumerate}[label=(\alph*)]
\item  THING
\item   Truth Table \begin{center}
    \begin{tabular}{c c c | c}
    I0 & I1 & A & Q \\ \hline
    0  & 0  & 0 & 0\\ 
    1  & 1  & 0 & 1\\
    0  & 1  & 1 & 0\\
    1  & 0  & 1 & 1\\
    \end{tabular}
    \end{center}
\item   This is a MUX!
\end{enumerate}
\end{solution}

\begin{figure}[H]
\centering
\includegraphics[width=0.6\textwidth]{figs/circuit.png}
\caption{Circuit 2 for parts (d) (e)}
\end{figure}
\item Write out and simplify the Boolean expression for circuit 2.

\item Write out the full truth table for circuit 2 based on the Boolean expression you found in part (d).

\item What inputs make the output of circuit 2 always 1?

\item What inputs make the output of circuit 2 always 0?

\end{enumerate}

\begin{solution}
\begin{enumerate}[label=(\alph*)]
\item  THING
\item   Truth Table \begin{center}
    \begin{tabular}{c c c c | c}
    A & B & C & D & output\\ \hline
    0 & 0 & 0 & D & output\\ 
    0 & 1 & 1 & D & output \\
    1 & 0 & 1 & D & output \\ 
    1 & 1 & 0 & D & output \\
    \end{tabular}
    \end{center}
\item   inputs
\item   inputs
\end{enumerate}
\end{solution}

\section*{Problem 3: Verilog}
For each example, identify the error in the Verilog code and suggest a fix. You don't have to rewrite the entire Verilog unless you think that's the most succinct \& clear way to answer.

\begin{minted}{verilog}
module example_one(
	input [1:0] a,
	input b, c,
	output x
);
	always @(*) begin
	    case (a)
	        2'b00 : x = b;
	        2'b01 : x = c;
	        2'b11 : x = b & c;
	        2'b10 : x = b | c;
	    endcase
	end
endmodule
\end{minted}

\begin{solution}
output x should be reg x. x is being assigned within an always block. 
\end{solution}

\begin{minted}{verilog}
module example_two(
	input a, b, c,
	output reg [1:0] x
);
    always @(*) begin
    	if (a & b & c) begin
    		x = 3;
    	end
    	else if (a & b) begin
    		x = 2;
    	end
    	else if (c) begin
    		x = 1;
    	end
    end
endmodule
\end{minted}

\begin{solution}
Include an else case to catch all other cases or 
\end{solution}

\item \begin{minted}{verilog}
module example_three(
	input [1:0] a,
	input toggle, sel, 
	output reg x
);
    always @(toggle) begin
        if (sel) begin
            x = a[1];
        end
        else if (!sel) begin
    	x = a[0];
        end
    end
endmodule
\end{minted}
\begin{solution}
just talk
\end{solution}

\section*{Problem 4: Circuit Drawing! (251 Only)}
Draw a circuit that can realize an arbitrary 2 input function of signals A and B. 

https://github.com/EECS150/eecs151_teaching/blob/master/homework/sp18/hw1/hw1_solution.pdf

https://github.com/EECS150/eecs151_teaching/blob/master/homework/sp18/hw1/hw1_solution.pdf

\pagebreak
\section*{Problem 5: Simplify, Simplify, Simplify (251 Only)}
Simplify the following expressions:
https://math.stackexchange.com/questions/2354936/complex-boolean-simplification
\begin{enumerate}[label=(\alph*)]
\item $(A + B)(A + C)$
\vspace{60mm}
\item $\overline{(A + B)}(A + \overline{B}C)\overline{(A + CB)}$
\vspace{60mm}
\item ($(\overline{A})(\overline{B})(\overline{C}) + \overline{A})(\overline{B})(C) + \overline{A}B\overline{C} + (A)(\overline{B})(\overline{C}) + (A)(\overline{B})C$
\end{enumerate}

\section*{Problem 5: Simplify, Simplify, Simplify (251 Only)}
Simplify the following expressions:

\begin{enumerate}[label=(\alph*)]
\item $(A + B)(A + C)$

\begin{solution}
$(A + B)(A + C)$
\end{solution}

\item $\overline{(A + B)}(A + \overline{B}C)\overline{(A + CB)}$

\begin{solution}
$\overline{(A + B)}(A + \overline{B}C)\overline{(A + CB)}$
\end{solution}

\item $(\overline{A})(\overline{B})(\overline{C}) + (\overline{A})(\overline{B})(C) + (\overline{A})(B)(\overline{C}) + (A)(\overline{B})(\overline{C}) + (A)(\overline{B})(C)$

\begin{solution}
$(\overline{A})(\overline{B})(\overline{C}) + (\overline{A})(\overline{B})(C) + (\overline{A})(B)(\overline{C}) + (A)(\overline{B})(\overline{C}) + (A)(\overline{B})(C)$
\end{solution}

\end{enumerate}


\section*{Problem 2: What's that Circuit!}
\begin{enumerate}[label=(\alph*)]


\begin{figure}[H]
\centering
\includegraphics[width=0.6\textwidth]{figs/mux.png}
\end{figure}

\item Write out and simplify the Boolean expression for the circuit above.

\vspace{40mm}

\item Write out the full truth table for the circuit above based on the Boolean expression you found in part (a).

\vspace{50mm}

\item What is the circuit above called?
\end{enumerate}
\pagebreak


\section*{Problem 2: Circuit Drawing!}
Draw a circuit that can realize any arbitrary 2 input function of signals A and B. Provide a truth table as well. You are allowed to use other signals beyond A and B.

\end{document}

